%-------------------------------------------------------------------------------
%	SECTION TITLE
%-------------------------------------------------------------------------------
% \newpage
\cvsection{Publications}


%-------------------------------------------------------------------------------
%	CONTENT
%-------------------------------------------------------------------------------
\begin{cventries}
	% %---------------------------------------------------------
	\cventrysix
	{PhD Thesis}
	{A Panorama on Classical Cryptography}
	{Nijmegen, The Netherlands} % Location
	{Dec. 2021} % Date(s)
	{Designing, Implementing, Breaking, Verifying, and Standardizing Cryptography} % Title
	{
		\begin{cvitems} % Description(s)
			\item {In this thesis we cover a large part of the classical cryptography world: we examine the design of new symmetric primitive; we explore implementation strategies of lightweight schemes; we analyze a new high performance algorithm; we use formal verification to prove the correctness of Elliptic Curve Cryptography implementations; and finally we describe one of the way algorithms are standardized.}
		\end{cvitems}
	}	% %---------------------------------------------------------

	\cventry
	{34th IEEE Computer Security Foundations Symposium}
	{A Coq proof of the correctness of X25519 in TweetNaCl} % Title
	{Dubrovnik, Croatia} % Location
	{Jun. 2021} % Date(s)
	{
		\begin{cvitems} % Description(s)
			\item {We formally prove that the C implementation of the X25519
			            key-exchange protocol in the TweetNaCl library correctly implements the protocol from Bernstein’s 2006 paper, as standardized in RFC 7748, as well
			            as the absence of undefined behavior. We also formally prove that X25519 is mathematically correct, i.e., that it correctly computes scalar multiplica-
			            tion on the elliptic curve Curve25519. The proofs are all computer-verified using Coq.}
		\end{cvitems}
	}

	% %---------------------------------------------------------
	\cventry
	{Cryptology and Network Security}
	{Assembly or Optimized C for Lightweight Cryptography on RISC-V?} % Title
	{Vienna, Austria} % Location
	{Dec. 2020} % Date(s)
	{
		\begin{cvitems} % Description(s)
			\item {In this work, we studied the general impact of optimizing symmetric-key algorithms in assembly and in plain C on RISC-V architectures. Additionally, we present optimized implementations of NIST's lightweight candidates, with speed-ups of up to 81\% over available implementations, and discuss general implementation strategies.}
		\end{cvitems}
	}

	\vspace{1em}
	% %---------------------------------------------------------
	\cventry
	{Advances in Cryptology – ASIACRYPT 2018, LNCS}
	{Cryptanalysis of MORUS} % Title
	{Brisbane, Australia} % Location
	{Dec. 2018} % Date(s)
	{
		\begin{cvitems} % Description(s)
			\item {We present a linear correlation in the keystream of full MORUS, which can be used to distinguish its output from random and to recover some plaintext bits in the broadcast setting.}
		\end{cvitems}
	}

	% %---------------------------------------------------------
	\cventry
	{Applied Cryptography and Network Security – ACNS 2018, LNCS}
	{KangarooTwelve: fast hashing based on Keccak-p} % Title
	{Leuven, Belgium} % Location
	{July 2018} % Date(s)
	{
		\begin{cvitems} % Description(s)
			\item {KangarooTwelve, a fast and secure arbitrary output-length hash function aiming at a higher speed than the FIPS 202’s SHA-3 and SHAKE functions.}
		\end{cvitems}
	}

	% %---------------------------------------------------------
	\cventry
	% {Daniel J. Bernstein, Stefan Kölbl, Stefan Lucks, Pedro Maat Costa Massolino, Florian Mendel, Kashif Nawaz, Tobias Schneider, Peter Schwabe, François-Xavier Standaert, Yosuke Todo, and Benoît Viguier} % Role
	{Cryptographic Hardware and Embedded Systems – CHES 2017, LNCS}
	{Gimli: A Cross-Platform Permutation} % Title
	{Taipei, Taiwan} % Location
	{Sept. 2017} % Date(s)
	{
		\begin{cvitems} % Description(s)
			\item {Gimli, a 384-bit permutation designed to achieve high security with high performance across a broad range of platforms.}
		\end{cvitems}
	}

	% %---------------------------------------------------------
	%   \cventry
	%     {Undergraduate Student Reporter} % Role
	%     {AhnLab} % Title
	%     {S.Korea} % Location
	%     {Oct. 2012 - Jul. 2013} % Date(s)
	%     {
	%       \begin{cvitems} % Description(s)
	%         \item {Drafted reports about IT trends and Security issues on AhnLab Company magazine.}
	%       \end{cvitems}
	%     }

	% %---------------------------------------------------------
\end{cventries}
